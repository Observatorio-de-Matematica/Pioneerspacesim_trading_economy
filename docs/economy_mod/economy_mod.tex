\documentclass[]{article}
\usepackage{amsmath}
\usepackage{amsfonts}
\usepackage{amssymb}
\usepackage [] {hyperref}
\usepackage{graphicx}
\usepackage{refcount}
\usepackage{fontawesome5}
\usepackage{enumitem}
\usepackage{ulem}
\begin{document}
	\paragraph{About economy\_mod.zip}
	Easily from equation 14, equation 13 is transformed to
	\begin{equation}\label{eq:profitRatio_ij}
		\dfrac{Profit_{j}}{Profit_{i}}=\Big(\dfrac{P_{i}}{P_{j}}\Big)^{a-1}=\Big(\dfrac{V^{**}_{ship_{j}}}{V^{**}_{ship_{i}}}\Big)^{\dfrac{a-1}{a}}
	\end{equation}
\paragraph{We pretend to have one merchant ship with its upgrades.} In each one we equip autopilot, default hyperdrive and some hydrogen for hyper jumps\footnote{Nothing else is needed for trading.}. We will take into consideration 16 ships, so, we assume, our ship  to have 15 upgrades.
Because pioneer lucks the upgrade mechanism, we will eventual call 
\begin{itemize}
\item Merchant\_15 free 3069t	dsminer -820
\item Merchant\_14 	lodos -445
\item Merchant\_13 	vatakara -817
\item Merchant\_12	nerodia -646
\item Merchant\_11	malabar -820
\item Merchant\_10	storeria -400
\item Merchant\_9	venturestar -261
\item Merchant\_8	molaramsayi -146
\item Merchant\_7	ac33 -240
\item Merchant\_6 	bluenose -57
\item Merchant\_5	deneb -62
\item Merchant\_4	skipjack -75
\item Merchant\_3	natrix -15
\item Merchant\_2	molamola -15
\item Merchant\_1	sinonatrix -15
\item Merchant\_0 free 13t	lunarshuttle -9
\end{itemize}
where minus numbers is the weight of referenced equipment plus a minimum jump fuel.
We want each ship's upgrade to give, eye detectable, more profit from previous upgrade. I consider 25\% such an improvement. So, \[1.25^{15}=28.42=\frac{Profit_{dsminer}}{Profit_{lunarshuttle}}=\Big(\dfrac{V^{**}_{ship_{dsiner}}}{V^{**}_{ship_{lunashuttle}}}\Big)^{\dfrac{a-1}{a}}=\Big(\dfrac{3069}{13}\Big)^{\dfrac{a-1}{a}}\Rightarrow\]\[a=2.581079957283195\]
Since we want 'lunarshuttle' and 'dsminer' ship's cargo fixed, $a$ is fixed because it is determined by them.
Only first and last free cargo are taken graded as 13t and 3069, free cargo of other ships will be estimated. From one to next we have $V^{**}_{i+1}=1.25^{\tfrac{a}{a-1}}*V^{**}_{i}=1.43947*V^{**}_{i}$. We add the weight of equipment and we have the new free cargo space of ships
\begin{itemize}
	\item dsminer	3889
\item 	lodos	2577
	
\item 	vatakara	2298
\item 	nerodia	1675
	
\item 	malabar	1535
\item 	storeria	897
\item 	venturestar	606
	
\item 	molaramsayi	386
\item 	ac33	406
	
\item 	bluenose	173
\item 	deneb	142
\item 	skipjack	131
	
\item 	natrix	54
	
\item 	molamola	42
\item 	sinonatrix	34
\item 	lunarshuttle	22	
\end{itemize}
We order 22 legal commodities such that each next commodity to give more profit from previous one as $Profit_{i+1}=b*Profit_{i}$. We want of course the first and last commodities to be traded by the last and first ships so as to have the same ratio of profits.  We find $b=1.17279319635886$\footnote{Filled in commodities.ods, sheet my\_economy, cell D2}.
\subparagraph{$constant$ equals to stock produced on a commodity priced $1\,credit$.} Stock is calculated at SpaceStation.lua as integer random function. I don't want to exceed integer limits, so i set $constant=10^9$. Given that, we want 'precious metals' stock at $V^{**}_{precious\_metals}=13t$, so its price calculated to $1426 credits/t$. All above determine the prices of all commodities.
\paragraph{Solve equilibrium} for 'lunarshuttle' at a price of 35002 credits and 'dsminer' at double its current price $2*2676331=5352662$ credits\footnote{After the following calculation of durations, it seemed justified to increase them by doubling the price of our flag ship as player's cash target.}. I used 'economy\_mod\_equilibrium2.wxmx' and wxmaxima for that\footnote{The old economy\_mod\_equilibrium.wxmx used constant discounts as the following text describes.}. The lunarshuttle's profit $Profit_{lunarshuttle}=2832.939$ was calculated my the help of equation 19, with those $a$ and $constant$ that just found.
\subparagraph*{Using found prices for ships,} the duration of economy game is $787su$, while, in current state, it last $97su$. We constructed our economy such $Profit_{ship_{i}}=Profit_{ship_{0}}*1.25^{i}$. If we use a constant discount $\lambda$ like the one at end of page 48 
\[\underset{i > current}{Ship_{i}value}=\lambda*T*(1-\dfrac{q_{current}}{q_{i}})+\dfrac{B_{current}}{2}\]
we observe that the duration of having each $ship_{i-1}$ is $1.25^i$ times less than the duration of 'lunarshuttle'. I want that duration $c_t$ constant. So, i have to apply different discount $\lambda_i$ for each ship.
Then the duration is given by \[duration=14*c_t +\dfrac{T}{q_{0}*1.25^{14}}\] I set $duration=3*current\;duration=3*97=291su$\footnote{If we set duration too high, $ship_{14}$ will cost more than 'dsminer' or worse $ship_{14}$ would be advertised at a price greater than its equilibrium value. Too low seems easy. For example if we set duration equal to current $97su$ then each ship is upgraded after $2su$, while, by $291su$, each upgrade of ship lasts $16su$.} So, $c_{t}= 16su$ We solve for $\lambda_i$ the
\[\lambda_i*\dfrac{0.20*T}{q_0*1.25^{i-1}}=c_t\] and  find all discounted ship prices.
\subparagraph{Ship prices and cargo.}
\begin{enumerate}[label=\Roman*,start=0]
\item 	Simple 35002 credits lunarshuttle	22t
	\item 46216 credits	sinonatrix	34t
		\item 85544 credits	molamola	42t		
		\item 124288 credits	natrix	54t
	\item 166210  credits	skipjack	131t
	\item 214543 credits	deneb	142t			
	\item 272416 credits	bluenose	173t		
	\item 343169  credits	ac33	406t
\item 430617 credits	molaramsayi	386t	
	\item 539305  credits	venturestar	606t
	\item 674779 credits	storeria	897t	
	\item 843877 credits	malabar	1535t
	\item  1055099 credits	nerodia	1675t
	\item  1319032 credits	vatakara	2298t				
	\item 1648889 credits	lodos	2577t			
	\item Crown ship 5352662 credits dsminer	3899t
\end{enumerate}

We set cargo field in json ship file and, by same amount used for its alteration, we alter capacity field too.

\paragraph{Legal commodity prices}
\begin{enumerate}
\item 	carbon ore	172
\item 	liquid oxygen	190
\item 	fertilizer	210
\item 	plastics	232
\item 	grain	257
\item 	metal alloys	284
\item 	metal ore	314
\item 	military fuel	348
\item 	textiles	385
\item 	farm machinery	425
\item 	fruit and veg	471
\item 	industrial machinery	521
\item 	animal meat	576
\item 	air processors	637
\item 	consumer goods	704
\item 	mining machinery	779
\item 	liquor	862
\item 	computers	953
\item 	medicines	1054
\item 	robots	1166
\item 	live animals	1290
\item 	precious metals	1426	
\end{enumerate}
\paragraph{Illegal commodity prices}
They are set near to commodities they are in current state and as prices they get the price of corresponding legal commodity. 
\begin{enumerate}
	\item hand weapons 704
	\item narcotics 953
	\item battle weapons 1054
	\item nerve gas 1166
	\item slaves 1290
\end{enumerate}
It seems that pioneer take care about their fluctuation.
\paragraph{A prerequisite mechanism is added} just before and in 'addRandomShipAdvert' function, in SpaceStaion.lua. By this, one can acquire a merchant ship only if he is its predecessor\footnote{An addition of a prerequisite tag in json ship files would be much better and cleaner solution but seemed to me that C code should be needed to export it to lua. So, i used a 'dirty' sequence of \textsf{'if'} statements in lua.}. If the predecessor of a ship is empty string then it can be acquired anyway. With the addition of such a tag keeping it empty will not affect pioneer at all. But ship designers can create a series of ships consisting of a model and its upgrades, set, only on upgrades, their prerequisite tag, solve for this series equilibrium and offer a fully calibrated on economy model.
\footnote{If prerequisite feature is not wished, i removed its code from SpaceStation.lua. Its relevant zip mod is named as no\_prerequisite\_mod.zip.}. 
\paragraph{UpdateEquipmentStock} function is altered a little so as major exports from a system to provide 36\% more profit for 6 months, from the player's very first visit, than that it will offer later. This is done because stations hope to lure new customers away from their routine trading. I hope to lure player to discover new systems by increased profit.
\paragraph{Taxi and assasination mission} have reduced their typical rewards according to expected average profit, when player trades in its full extend\footnote{Advertisers understand that they have to pay extra and compensate for the loss of big ships, when they want to take them out of their usual profitable business.}. I wanted initial trading not to be overlapped by mission profits. For taxi mission their average profit set to 2/3 of trading profit and for assassination missions to 3/4 of the same profit, where trading profit is taken at least equal to that of 'lunarshuttle'.
\end{document}